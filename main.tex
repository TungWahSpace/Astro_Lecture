\documentclass[a4paper,10pt,twoside,openany]{ctexbook}

\usepackage{imakeidx}
\usepackage{siunitx}
\usepackage{multicol}
\usepackage[
    left=2cm,right=2cm,
    top=3cm,bottom=2.5cm
]{geometry}
\usepackage{subfiles}
\usepackage{titletoc}
% \titlecontents{section}
\setcounter{tocdepth}{3}
\usepackage{hyperref}
% \usepackage[utf8]{inputenc}
% \usepackage[russian]{babel}
% \usepackage[T2A]{fontenc}

\usepackage{fontspec}
\setmainfont{CMU Serif}

\usepackage[titles]{tocloft}
\renewcommand{\cftdot}{$\cdot$}
\renewcommand{\cftdotsep}{1.5}
\setlength{\cftbeforechapskip}{10pt}

\renewcommand{\cftchapleader}{\cftdotfill{\cftchapdotsep}}
\renewcommand{\cftchapdotsep}{\cftdotsep}
\makeatletter
\renewcommand{\numberline}[1]{%
\settowidth\@tempdimb{#1\hspace{0.5em}}%
\ifdim\@tempdima<\@tempdimb%
  \@tempdima=\@tempdimb%
\fi%
\hb@xt@\@tempdima{\@cftbsnum #1\@cftasnum\hfil}\@cftasnumb}
\makeatother

\renewcommand{\contentsname}{目\hspace{1em}录}
\newcommand{\introduction}[1]{
    \begin{quotation}
        \kaishu#1
    \end{quotation}
}
\newcommand{\Text}[1]{
    \songti
    \begin{multicols*}{2}
        #1
    \end{multicols*}
}


\begin{document}
    \begin{titlepage}
        \subfile{title}
    \end{titlepage}
    \newpage
    \begin{multicols*}{2}
        \begin{center}
            \tableofcontents
            \thispagestyle{empty}
        \end{center}
    \end{multicols*}
    
    
    \pagenumbering{arabic}    
    
    \include{chapters/introduction}
    
    \part{观测}{
        \include{chapters/obs_introduction}
        \include{chapters/sphere}
        \include{chapters/obs_guide}
        \include{chapters/surface_obs}
        \include{chapters/instrument}
        \include{chapters/data}
    }

    \part{天体物理学}{
        \include{chapters/orbit}
        \include{chapters/radiation}
        \include{chapters/thermal}
    }

    \part{天体}{
        \include{chapters/solar_sys}
        \include{chapters/stella}
        \include{chapters/sun}
        \include{chapters/bin_stella}
        \include{chapters/var_stella}
        \include{chapters/hd_stella}
        \include{chapters/interstella}
        \include{chapters/cluster}
        \include{chapters/milky_way}
        \include{chapters/galaxy}
        \include{chapters/exoplanet}
    }

    \part{扩展}{
        \include{chapters/cosmos}
        \include{chapters/astrobiology}
    }
    
    
\end{document}